
\documentclass [a4paper,titlepage]{report}
%\pagestyle{plain}
\usepackage{makeidx}
\usepackage[spanish]{babel}
\usepackage[utf8x]{inputenc}
\usepackage[T1]{fontenc}
\usepackage{lmodern}
\usepackage{enumerate}
\usepackage{longtable}
\usepackage{graphicx}
\usepackage{ucs}
\usepackage{textcomp}
\usepackage{alltt}
\usepackage{listings}
\usepackage{moreverb}
\usepackage{upquote}

%\lstset{basicstyle=\footnotesize\ttfamily,showstringspaces=false,breaklines,frame=single, rulecolor=\color{ListingBorderColor}, xleftmargin=0cm, linewidth=0.95\textwidth}

\setlength{\parskip}{1ex plus 0.5ex minus 0.2ex}

\title{Analizador Léxico \\Compiladores e Intérpretes \\2012}

\author{Garay, I\~{n}aki LU 67387}

\makeindex

\begin{document}

\maketitle

\tableofcontents

\newpage

\section{Uso}

\subsection{Invocación}

Utilizando el intérprete de Python para correr el script del módulo principal:

\begin{verbatim}
    C:\> python.exe main.py <archivo_entrada> [<archivo_salida>]
\end{verbatim}

La especificación del archivo de salida es opcional. 
Si se omite, el analizador léxico mostrará su salida por pantalla.
Si no se especifica un archivo de entrada, o se pasan más argumentos que el 
archivo de entrada y el de salida, el analizador léxico mostrará un mensaje 
explicando el uso por linea de comando.

\subsection{Interpretación de Salida}

Mientras procesa el archivo, el analizador léxico mostrará en su salida los
tokens reconocidos en una tabla, con el siguiente formato:

\begin{verbatim}
$ python2 main.py ./tests/006-err-ampersand.java 
1:0     -   <INT_TYPE>      :: int
1:3     -   <SEPARATOR>     ::  
1:4     -   <IDENTIFIER>    :: fun
1:7     -   <PAREN_OPEN>    :: (
1:8     -   <INT_TYPE>      :: int
1:11    -   <SEPARATOR>     ::  
1:12    -   <IDENTIFIER>    :: a
1:13    -   <PAREN_CLOSE>   :: )
1:14    -   <SEPARATOR>     ::  
1:15    -   <BRACE_OPEN>    :: {
1:16    -   <SEPARATOR>     ::  
2:4     -   <IDENTIFIER>    :: un
2:6     -   <SEPARATOR>     ::  
ERROR: Line: 2, Col: 7 :: Token no reconocido.
In line 2:7
    un & amper;
-------^
\end{verbatim}

La primera columna de la tabla indica el número de linea y la columna en la cual
se encontró el token. La siguiente columna indica el tipo del token. La tercera
columna muestra el lexema del token.

\subsection{Bugs conocidos}
\begin{itemize}

\item El reporte de la columna en la cual se detecta un error léxico
no siempre es correcto.

\item El reporte de la linea en la cual se detecta un error léxico no 
es correcto si la linea anterior termina en espacios.

\item La detección del operador prohibido
\texttt{\textless{}\textless{}\textless{}=} no es correcto.

\end{itemize}

\section{Análisis Léxico}

\subsection{Alfabeto de entrada}

El alfabeto de entrada son todos los caracteres de la codificación ASCII.

\subsection{Definición de Tokens}

Las expresiones regulares se expresaron utilizando la sintaxis válida para JLex.

La siguiente tabla muestra los tokens reconocidos por el analizador léxico.

\begin{longtable}{l | l | l}
\bfseries{Token}     & \bfseries{Expresión Regular}                                          & \bfseries{Ejemplo}                \tabularnewline \endhead
TK\_IDENTIFIER       & [a-zA-Z\_{}\textbackslash{}\${}][a-zA-Z\_{}\textbackslash{}\${}0-9]*  & identifier                        \tabularnewline
TK\_INT\_LITERAL     & (0\textbar{}[1-9]([0-9])*)                                            & 42                                \tabularnewline
TK\_CHAR\_LITERAL    & (\textbackslash{}'[\^{}('\textbackslash{}\textbackslash{})]\textbackslash{}')\textbar{}(\textbackslash{}'[\textbackslash{}\textbackslash{}\textbackslash{}\textbackslash{}\textbar{}\textbackslash{}\textbackslash{}'\textbar{}\textbackslash{}\textbackslash{}\textbackslash{}"\textbar{}\textbackslash{}\textbackslash{}n]\textbackslash{}')  & \textquotesingle{}c\textquotesingle{}   \tabularnewline
TK\_STRING\_LITERAL  & (\textbackslash{}"\textbackslash{}"\textbar{}\textbackslash{}"([\^{}(\textbackslash{}"\textbackslash{}\textbackslash{})]\textbar{}[\textbackslash{}\textbackslash{}\textbackslash{}\textbackslash{}\textbar{}\textbackslash{}\textbackslash{}'\textbar{}\textbackslash{}\textbackslash{}\textbackslash{}"\textbar{}\textbackslash{}\textbackslash{}n])*\textbackslash{}") & \textquotedbl{}s\textquotedbl{}   \tabularnewline
TK\_BRACE\_OPEN      & \{{}                                                                  & \{{}                              \tabularnewline
TK\_BRACE\_CLOSE     & \}{}                                                                  & \}{}                              \tabularnewline
TK\_PAREN\_OPEN      & \textbackslash{})                                                     & (                                 \tabularnewline
TK\_PAREN\_CLOSE     & \textbackslash{}(                                                     & )                                 \tabularnewline
TK\_PERIOD           & .                                                                     & .                                 \tabularnewline
TK\_COMMA            & ,                                                                     & ,                                 \tabularnewline
TK\_SEMICOLON        & ;                                                                     & ;                                 \tabularnewline
TK\_ASSIGNMENT       & =                                                                     & =                                 \tabularnewline
TK\_ADD              & +                                                                     & +                                 \tabularnewline
TK\_SUB              & -                                                                     & -                                 \tabularnewline
TK\_MUL              & *                                                                     & *                                 \tabularnewline
TK\_DIV              & \textbackslash{}/                                                     & /                                 \tabularnewline
TK\_MOD              & \%{}                                                                  & \%{}                              \tabularnewline
TK\_LT               & \textless{}                                                           & \textless{}                       \tabularnewline
TK\_GT               & \textgreater{}                                                        & \textgreater{}                    \tabularnewline
TK\_LTEQ             & \textless{}=                                                          & \textless{} =                     \tabularnewline
TK\_GTEQ             & \textgreater{}=                                                       & \textgreater{} =                  \tabularnewline
TK\_EQUALS           & ==                                                                    & ==                                \tabularnewline
TK\_NOTEQUALS        & !=                                                                    & !=                                \tabularnewline
TK\_NOT              & !                                                                     & !                                 \tabularnewline
TK\_AND              & \&{}\&{}                                                              & \&{}\&{}                          \tabularnewline
TK\_OR               & \textbar{}\textbar{}                                                  & \textbar{}\textbar{}              \tabularnewline
TK\_BOOLEAN          & boolean                                                               & boolean                           \tabularnewline
TK\_CHAR             & char                                                                  & char                              \tabularnewline
TK\_CLASS            & class                                                                 & class                             \tabularnewline
TK\_CLASSDEF         & classDef                                                              & classDef                          \tabularnewline
TK\_ELSE             & else                                                                  & else                              \tabularnewline
TK\_EXTENDS          & extends                                                               & extends                           \tabularnewline
TK\_FALSE            & false                                                                 & false                             \tabularnewline
TK\_FOR              & for                                                                   & for                               \tabularnewline
TK\_IF               & if                                                                    & if                                \tabularnewline
TK\_INT              & int                                                                   & int                               \tabularnewline
TK\_NEW              & new                                                                   & new                               \tabularnewline
TK\_NULL             & null                                                                  & null                              \tabularnewline
TK\_RETURN           & return                                                                & return                            \tabularnewline
TK\_STRING           & String                                                                & String                            \tabularnewline
TK\_SUPER            & super                                                                 & super                             \tabularnewline
TK\_THIS             & this                                                                  & this                              \tabularnewline
TK\_TRUE             & true                                                                  & true                              \tabularnewline
TK\_VOID             & void                                                                  & void                              \tabularnewline
TK\_WHILE            & while                                                                 & while                             \tabularnewline
TK\_EOF              & \textless{}EOF\textgreater{}                                          & \textless{}EOF\textgreater{}      \tabularnewline
\end{longtable}

% \noindent
% Char literal regular expression:

% \begin{verbatim}
%     (
%         \'
%         [ ^(  \'  \\  ) ]
%         \'
%     ) 
% | 
%     (
%         \'  
%         [  \\  \\  
%         |  \\  \'  
%         |  \\  \"  
%         |  \\n  
%         ]
%         \'  
%     )
% \end{verbatim}

% \noindent
% String literal regular expression:

% \begin{verbatim}
% (
%     \"
%     \"
% |  
%     \"
%     (   
%             [  ^(  \"  \\  )  ]
%         |  
%             [  \\  \\  
%             |  \\  \'
%             |  \\  \"
%             |  \\  n
%             ]
%     )*
%     \"
% )
% \end{verbatim}

\subsection{Operadores Prohibidos}

\begin{longtable}{l | l}
Operador                                    & Expresión Regular                             \tabularnewline \endhead
\textasciitilde{}                           & \textasciitilde{}                             \tabularnewline
?                                           & ?                                             \tabularnewline
:                                           & :                                             \tabularnewline
++                                          & ++                                            \tabularnewline
--                                          & --                                            \tabularnewline
\&{}                                        & \&{}                                          \tabularnewline
\textbar{}                                  & \textbar{}                                    \tabularnewline
\^{}                                        & \^{}                                          \tabularnewline
\textless{}\textless{}                      & \textless{}\textless{}                        \tabularnewline
\textgreater{}\textgreater{}                & \textgreater{}\textgreater{}                  \tabularnewline
\textgreater{}\textgreater{}\textgreater{}  & \textgreater{}\textgreater{}\textgreater{}    \tabularnewline
+=                                          & +=                                            \tabularnewline
-=                                          & -=                                            \tabularnewline
*=                                          & *=                                            \tabularnewline
/=                                          & /=                                            \tabularnewline
\&{}=                                       & \&{}=                                         \tabularnewline
\textbar{}=                                 & \textbar{}=                                   \tabularnewline
\^{}=                                       & \^{}=                                         \tabularnewline
\%{}=                                       & \%{}=                                         \tabularnewline
\textless{}\textless{}=                     & \textless{}\textless{}=                       \tabularnewline
\textgreater{}\textgreater{}=               & \textgreater{}\textgreater{}=                 \tabularnewline
\textgreater{}\textgreater{}\textgreater{}= & \textgreater{}\textgreater{}\textgreater{}=   \tabularnewline
\end{longtable}\tabularnewline

\subsection{Palabras Prohibidas}

\begin{verbatim}
    abstract       interface
    break          
    byte           long
                   native
    byvalue        none
    case           operator
    cast           outer
    catch          package
    const          private
    continue       protected
    default        public
    do             rest
    double         
    final          short
    finally        static
    float          switch
    future         synchronized
    generic        throw
    goto           throws
    implements     transient
    import         try
    inner          var
    instanceof     volatile
\end{verbatim}

\section{Gramática}

\subsection{Gramatica original}

\begin{verbatim}
<start>                 ::= <class>+

<class>                 ::= class <identifier> <super>? 
                            { <field>* <ctor>* <method>* }

<super>                 ::= extends <identifier>

<field>                 ::= <type> <var_declarator_list> ;

<method>                ::= <method_type> <identifier> 
                            <formal_args> <block>

<ctor>                  ::= <identifier> <formal_args> <block>

<formal_args>           ::= ( <formal_arg_list>? )

<formal_arg_list>       ::= <formal_arg>
<formal_arg_list>       ::= <formal_arg> , <formal_arg_list>

<formal_arg>            ::= <type> <identifier>

<method_type>           ::= void
<method_type>           ::= <type>

<type>                  ::= <primitive_type>
<type>                  ::= <identifier>

<primitive_type>        ::= boolean
<primitive_type>        ::= char
<primitive_type>        ::= int
<primitive_type>        ::= String

<var_declarator_list>   ::= <identifier> , <var_declarator_list>
<var_declarator_list>   ::= <identifier>

<block>                 ::= { <statement>* }

<statement>             ::= ;
<statement>             ::= if ( <expression> ) <statement>
<statement>             ::= if ( <expression> ) <statement> 
                            else <statement>
<statement>             ::= return <expression>? ;
<statement>             ::= <block>

<expressions>           ::= <expression> <binary_op> <expression>
<expressions>           ::= <unary_op> <expression>
<expressions>           ::= <primary>

<binary_op>             ::= =
<binary_op>             ::= || 
<binary_op>             ::= &&
<binary_op>             ::= ==
<binary_op>             ::= !=
<binary_op>             ::= <
<binary_op>             ::= >
<binary_op>             ::= <=
<binary_op>             ::= >=
<binary_op>             ::= +
<binary_op>             ::= -
<binary_op>             ::= *
<binary_op>             ::= /
<binary_op>             ::= %

<unary_op>              ::= !
<unary_op>              ::= +
<unary_op>              ::= -

<primary>               ::= <new_expr>
<primery>               ::= <identifier>

<new_expr>              ::= <literal>
<new_expr>              ::= this
<new_expr>              ::= this . <identifier>
<new_expr>              ::= ( <expression> )
<new_expr>              ::= new <identifier> 
                            <actual_args>
<new_expr>              ::= <identifier> <actual_args>
<new_expr>              ::= <primary> . <identifier> 
                            <actual_args>
<new_expr>              ::= super . <identifier> 
                            <actual_args>

<literal>               ::= null
<literal>               ::= true
<literal>               ::= false
<literal>               ::= <int_literal>
<literal>               ::= <char_literal>
<literal>               ::= <string_literal>

<actual_args>           ::= ( <expr_list>? )

<expr_list>             ::= <expression>
<expr_list>             ::= <expression> , <expr_list>

\end{verbatim}

\subsection{Gramatica modificada}

Cambios realizados:
\begin{itemize}

\item Producciones para \texttt{classDef}

\item Producciones para expresiones, tomando en cuenta precedencia y 
asociatividad.

\item Producciones para sentencia \texttt{while}.

\item Producciones para sentencia \texttt{for}.
\end{itemize}

\begin{verbatim}
<start>                 ::= <classdef>+ <class>+

<classdef>              ::= classdef <identifier> <super>? 
                            { <field>* <classdef_ctor>* 
                            <classdef_method>* }

<class>                 ::= class <identifier> <super>? 
                            { <field>* <ctor>* <method>* }

<super>                 ::= extends <identifier>

<field>                 ::= <type> <var_declarator_list> ;

<classdef_method>       ::= <method_type> <identifier> 
                            <formal_args>

<method>                ::= <method_type> <identifier> 
                            <formal_args> <block>

<classdef_ctor>         ::= <identifier> <formal_args>

<ctor>                  ::= <identifier> <formal_args> <block>

<formal_args>           ::= ( <formal_arg_list>? )

<formal_arg_list>       ::= <formal_arg>
<formal_arg_list>       ::= <formal_arg> , <formal_arg_list>

<formal_arg>            ::= <type> <identifier>

<method_type>           ::= void
<method_type>           ::= <type>

<type>                  ::= <primitive_type>
<type>                  ::= <identifier>

<primitive_type>        ::= boolean
<primitive_type>        ::= char
<primitive_type>        ::= int
<primitive_type>        ::= String

<var_declarator_list>   ::= <identifier> , <var_declarator_list>
<var_declarator_list>   ::= <identifier>

<block>                 ::= { <statement>* }

<statement>             ::= ;
<statement>             ::= if ( <expression> ) <statement>
<statement>             ::= if ( <expression> ) <statement> 
                            else <statement>
<statement>             ::= while ( <statement> ) <statement>
<statement>             ::= for ( <statement> ; <statement> ; 
                            <statement> ) <statement>
<statement>             ::= return <expression>? ;
<statement>             ::= <block>

<expression>            ::= <assignment_expr>

<assignment_expr>       ::= <logical_expr>

<logical_expr>          ::= <logical_or_expr> <logical_expr_rest>

<logical_expr_rest>     ::= LAMBDA
<logical_expr_rest>     ::= ASSIGNMENT <logical_expr>

<logical_or_expr>       ::= <logical_and_expr> <logical_or_expr_rest>

<logical_or_expr_rest>  ::= LAMBDA
<logical_or_expr_rest>  ::= OR <logical_or_expr>

<logical_and_expr>      ::= <equality_expr> <logical_and_expr_rest>

<logical_and_expr_rest> ::= LAMBDA
<logical_and_expr_rest> ::= AND <logical_and_expr>

<equality_expr>         ::= <relational_expr> <equality_expr_rest>

<equality_expr_rest>    ::= LAMBDA
<equality_expr_rest>    ::= EQUALS <equality_expr>
<equality_expr_rest>    ::= NOTEQUALS <equality_expr>

<relational_expr>       ::= <term_expr> <relational_expr_rest>

<relational_expr_rest>  ::= LAMBDA
<relational_expr_rest>  ::= LT <relational_expr>
<relational_expr_rest>  ::= GT <relational_expr>
<relational_expr_rest>  ::= LTEQ <relational_expr>
<relational_expr_rest>  ::= GTEQ <relational_expr>

<term_expr>             ::= <factor_expr> <additive_expr_rest>

<additive_expr_rest>    ::= LAMBDA
<additive_expr_rest>    ::= ADD <term_expr>
<additive_expr_rest>    ::= SUB <term_expr>

<factor_expr>           ::= <unary_expr> <multiplicative_expr_rest>

<factor_expr_rest>      ::= LAMBDA
<factor_expr_rest>      ::= MUL <factor_expr>
<factor_expr_rest>      ::= DIV <factor_expr>
<factor_expr_rest>      ::= MOD <factor_expr>

<unary_expr>            ::= ADD <unary_expr>
<unary_expr>            ::= SUB <unary_expr>
<unary_expr>            ::= NOT <unary_expr>
<unary_expr>            ::= <primary>

<primary>               ::= <new_expr>
<primery>               ::= <identifier>

<new_expr>              ::= <literal>
<new_expr>              ::= this
<new_expr>              ::= this . <identifier>
<new_expr>              ::= ( <expression> )
<new_expr>              ::= new <identifier> <actual_args>
<new_expr>              ::= <identifier> <actual_args>
<new_expr>              ::= <primary> . <identifier> <actual_args>
<new_expr>              ::= super . <identifier> <actual_args>

<literal>               ::= null
<literal>               ::= true
<literal>               ::= false
<literal>               ::= <int_literal>
<literal>               ::= <char_literal>
<literal>               ::= <string_literal\textless{}

<actual_args>           ::= ( \textless{}expr_list>? )

<expr_list>             ::= <expression\textless{}
<expr_list>             ::= <expression> , <expr_list>
\end{verbatim}\textless{}
\textless{}subsection{Gramatica modificada 2}

La siguiente gramatica es identica a la anterior con los siguientes 
cambios:

\begin{itemize}

\item Los simbolos terminales has sido reemplazados por los tokens 
producidos por el scanner.

\item Se eliminaron las extensiones en la gramatica, eliminando el 
uso de los +, *, ?.

\item Se agrego una produccion de \textless{}start\textgreater{} para 
eliminar el +.

\item Se agrego una produccion de \textless{}class\textgreater{} para 
eliminar el \textless{}super\textgreater{}?

\item Se agrego una produccion \textless{}class\_body\textgreater{} 
para factorizar { field ctor method }.

\item Se agregaron las producciones \textless{}fields\textgreater{} 
\textless{}ctors\textgreater{} y \textless{}methods\textgreater{} 
para eliminar el \textless{}field\textgreater{}*.

\item Se agrego un simbolo \textless{}statements\textgreater{} 
para el \textless{}statement\textgreater{}* en la produccion de 
\textless{}block\textgreater{}.

\item Se agrego una produccion de \textless{}statement\textgreater{} 
para el return \textless{}expression\textgreater{}? ;

\item agregada produccion \textless{}actual\_args\textgreater{} para 
\textless{}expr\_list\textgreater{}?.

\end{itemize}

\begin{verbatim}
<start>                 ::= <class>
<start>                 ::= <class> <start>

<class>                 ::= TK_CLASS TK_IDENTIFIER <class_body>
<class>                 ::= TK_CLASS TK_IDENTIFIER <super> <class_body>

<class_body>            ::= TK_BRACE_OPEN 
                            <fields> <ctors> <methods> 
                            TK_BRACE_CLOSE

<fields>                ::= LAMBDA
<fields>                ::= <field>
<fields>                ::= <field> <fields>

<ctors>                 ::= LAMBDA
<ctors>                 ::= <ctor>
<ctors>                 ::= <ctor> <ctors>

<methods>               ::= LAMBDA
<methods>               ::= <method>
<methods>               ::= <method> <methods>

<super>                 ::= TK_EXTENDS TK_IDENTIFIER

<field>                 ::= <type> <var_declarator_list> TK_SEMICOLON

<method>                ::= <method_type> TK_IDENTIFIER 
                            <formal_args> <block>

<ctor>                  ::= TK_IDENTIFIER <formal_args> <block>

<formal_args>           ::= TK_PAREN_OPEN TK_PAREN_CLOSE
<formal_args>           ::= TK_PAREN_OPEN 
                            <formal_arg_list> 
                            TK_PAREN_CLOSE

<formal_arg_list>       ::= <formal_arg>
<formal_arg_list>       ::= <formal_arg> TK_COMMA <formal_arg_list>

<formal_arg>            ::= <type> TK_IDENTIFIER

<method_type>           ::= TK_VOID
<method_type>           ::= <type>

<type>                  ::= TK_IDENTIFIER
<type>                  ::= <primitive_type>

<primitive_type>        ::= TK_BOOLEAN
<primitive_type>        ::= TK_CHAR
<primitive_type>        ::= TK_INT
<primitive_type>        ::= TK_STRING

<var_declarator_list>   ::= TK_IDENTIFIER
<var_declarator_list>   ::= TK_IDENTIFIER TK_COMMA 
                            <var_declarator_list>

<block>                 ::= TK_BRACE_OPEN <statements> TK_BRACE_CLOSE

<statements>            ::= <statement>
<statements>            ::= <statement> <statements>

<statement>             ::= TK_SEMICOLON
<statement>             ::= TK_IF TK_PAREN_OPEN <expression> 
                            TK_PAREN_CLOSE <statement>
<statement>             ::= TK_IF 
                            TK_PAREN_OPEN <expression> 
                            TK_PAREN_CLOSE <statement> 
                            TK_ELSE <statement>
<statement>             ::= TK_RETURN TK_SEMICOLON
<statement>             ::= TK_RETURN <expression> TK_SEMICOLON
<statement>             ::= <block>
<statement>             ::= TK_FOR TK_PAREN_OPEN <statement> 
                            TK_SEMICOLON <statement> TK_SEMICOLON
                            <statement> TK_PAREN_CLOSE <statement>
<statement>             ::= TK_WHILE TK_PAREN_OPEN <statement> 
                            TK_PAREN_CLOSE <statement>

<expression>            ::= <assignment_expr>

<assignment_expr>       ::= <logical_expr>

<logical_expr>          ::= <logical_or_expr> <logical_expr_rest>

<logical_expr_rest>     ::= LAMBDA
<logical_expr_rest>     ::= TK_ASSIGNMENT <logical_expr>

<logical_or_expr>       ::= <logical_and_expr> <logical_or_expr_rest>

<logical_or_expr_rest>  ::= LAMBDA
<logical_or_expr_rest>  ::= TK_OR <logical_or_expr>

<logical_and_expr>      ::= <equality_expr> <logical_and_expr_rest>

<logical_and_expr_rest> ::= LAMBDA
<logical_and_expr_rest> ::= TK_AND <logical_and_expr>

<equality_expr>         ::= <relational_expr> <equality_expr_rest>

<equality_expr_rest>    ::= LAMBDA
<equality_expr_rest>    ::= TK_EQUALS <equality_expr>
<equality_expr_rest>    ::= TK_NOTEQUALS <equality_expr>

<relational_expr>       ::= <term_expr> <relational_expr_rest>

<relational_expr_rest>  ::= LAMBDA
<relational_expr_rest>  ::= TK_LT <relational_expr>
<relational_expr_rest>  ::= TK_GT <relational_expr>
<relational_expr_rest>  ::= TK_LTEQ <relational_expr>
<relational_expr_rest>  ::= TK_GTEQ <relational_expr>

<term_expr>             ::= <factor_expr> <additive_expr_rest>

<additive_expr_rest>    ::= LAMBDA
<additive_expr_rest>    ::= TK_ADD <term_expr>
<additive_expr_rest>    ::= TK_SUB <term_expr>

<factor_expr>           ::= <unary_expr> <multiplicative_expr_rest>

<factor_expr_rest>      ::= LAMBDA
<factor_expr_rest>      ::= TK_MUL <factor_expr>
<factor_expr_rest>      ::= TK_DIV <factor_expr>
<factor_expr_rest>      ::= TK_MOD <factor_expr>

<unary_expr>            ::= TK_ADD <unary_expr>
<unary_expr>            ::= TK_SUB <unary_expr>
<unary_expr>            ::= TK_NOT <unary_expr>
<unary_expr>            ::= <primary>

<primary>               ::= TK_IDENTIFIER
<primary>               ::= <new_expr>

<new_expr>              ::= <literal>
<new_expr>              ::= TK_THIS
<new_expr>              ::= TK_THIS TK_PERIOD TK_IDENTIFIER
<new_expr>              ::= TK_PAREN_OPEN <expression> TK_PAREN_CLOSE
<new_expr>              ::= TK_NEW TK_IDENTIFIER <actual_args>
<new_expr>              ::= TK_IDENTIFIER <actual_args>
<new_expr>              ::= <primary> TK_PERIOD 
                            TK_IDENTIFIER <actual_args>
<new_expr>              ::= TK_SUPER TK_PERIOD 
                            TK_IDENTIFIER <actual_args>

<literal>               ::= TK_NULL
<literal>               ::= TK_TRUE
<literal>               ::= TK_FALSE
<literal>               ::= TK_INT_LITERAL
<literal>               ::= TK_CHAR_LITERAL
<literal>               ::= TK_STRING_LITERAL

<actual_args>           ::= TK_PAREN_OPEN TK_PAREN_CLOSE
<actual_args>           ::= TK_PAREN_OPEN <expr_list> TK_PAREN_CLOSE

<expr_list>             ::= <expression>
<expr_list>             ::= <expression> TK_COMMA <expr_list>
\end{verbatim}

\section{Diseño}

La principal decisión de diseño que afecto la implementación fue la
abstracción de los estados del automata finito reconocedor.

El analizador léxico se desarrolló utilizando únicamente la versión
2.7 del lenguaje Python (+www.python.org+).

Para implementar el analizador léxico se realizó la especificación de
la máquina  de estados basándose en las expresiones regulares
definidas en la sección  anterior, para luego representarlo en código
Python.

\subsection{Archivo \texttt{main.py}}

Contiene el punto de inicio de ejecución del programa, instanciando
el \texttt{Scanner} con el archivo de entrada, y realiza checkeo de
errores en la invocacíón.

\subsection{Archivo \texttt{scanner.py}}

Contiene la definición de la clase \texttt{Scanner}. 

\subsection{Archivo \texttt{states.py}}

Contiene la definicion de la clase \texttt{State}, la instanciación de
los estados del autómata finito reconocedor y sus transiciones.

\subsection{Archivo \texttt{tokens.py}}

Contiene la definición de la clase \texttt{Token} y
\texttt{TokenType}, y la  instanciación de los tokens retornados por
el analizador léxico.

\subsection{Archivo \texttt{errors.py}}

Contiene la definición de la clase \texttt{LexicalError}.

\subsection{Clase \texttt{Scanner}}

\texttt{Scanner} es la clase que representa al analizador léxico propiamente.
Este implementa el método \texttt{get\_token()} que devuelve
secuencialmente todos los tokens reconocidos en un dado archivo que
contiene código \texttt{MiniJava-Decaf}. 

\subsection{Clase \texttt{State}}

La clase principal utilizada es la llamada \texttt{State}. Esta
representa una abstracción de un estado de un autómata
finito, que cuenta con una serie de funciones de checkeo para
determinar si se activa una transición o no.

\subsection{Clases \texttt{Token} y \texttt{TokenType}}

Para representar a los tokens se armaron dos clases:
\texttt{TokenType} y \texttt{Token}. \texttt{TokenType}, definida en
\texttt{tokens.py}, es una abstracción sobre los distintos tipos de
tokens reconocidos y especificados en la sección anterior.
\texttt{Token} abstrae un token y almacena los metadatos asociados a
el, y es instanciado según el análisis del archivo de código fuente
en cuestión.

\subsection{Clase \texttt{LexicalError}}

Para el manejo de errores, se creó el tipo de excepción
\texttt{LexicalError} dentro del archivo \texttt{errors.py}.

\section{Verificación}

\subsection{Errores detectados}

El analizador léxico reconoce los siguientes tipos de errores:

\begin{itemize}

\item Comentario del tipo /* */ que no esté propiamente cerrado.

\item Informa de tokens no reconocidos.

\item Caracter no reconocido: si se intenta ingresar un caracter que
no pertenece al alfabeto se producirá un error.

\item Si el archivo de entrada especificado no existe, se producirá un
error.

\item Si se encuentra un caracter literal de mas de un caracter, e.g.
\textquotesingle{}hola\textquotesingle{}, se producirá un error. En
cambio, si es un caracter válido, e.g.
\textquotesingle{}n\textquotesingle{}, se aceptará.

\end{itemize}

\subsection{Casos de prueba}

Cuando un error es detectado, también se muestra por pantalla la
ubicación de la porción del texto que presenta el error. Por ejemplo:

\begin{verbatim}
 ERROR: Line: 3, Col: 4 :: Comentario no cerrado.
 In line 3:4
     /* comment
 ----^
\end{verbatim}

\begin{verbatim}
 ERROR: Line: 2, Col: 10 :: Token no reconocido.
 In line 2:10
     a = 1 # 2
 ----------^
\end{verbatim}




\noindent
\textbf{Test \texttt{000-cor-tokens.java}}

\textbf{Descripción:} El archivo contiene todos los tokens reconocidos
por el scanner, uno por linea.

\textbf{Resultado esperado:} Caso correcto.




\noindent
\textbf{Test \texttt{001-err-forbidden-words.java}}

\textbf{Descripción:} El archivo contiene todas las palabras
prohibidas, una por linea, comentadas. El analizador léxico falla al
detectar la primera, se puede modificar el archivo descomentando las
palabras para probar la detección de cada una de las palabras.

\textbf{Resultado esperado:} Caso erróneo. detección correcta de palabras
prohibidas, emitiendo un error y un mensaje apropiado.




\noindent
\textbf{Test \texttt{002-err-forbidden-operator.java}}

\textbf{Descripción:} El archivo contiene todos los operadores
prohibidos, uno por linea, comentados. El analizador léxico falla al
detectar el primero; se puede modificar el archivo descomentando los
otros operadores para probar la detección de cada uno.

\textbf{Resultado esperado:} Caso erróneo. 




\noindent
\textbf{Test \texttt{003-cor-ejemplo.java}}

\textbf{Descripción:} Este es el caso de test principal para detección
correcta de tokens.

\textbf{Resultado esperado:} Caso correcto. Todos los tokens son
reconocidos con éxito según lo esperado, y los comentarios son
obviados sin modificar la posición final de los tokens alrededor de
ellos.




\noindent
\textbf{Test \texttt{004-err-hash.java}}

\textbf{Descripción:} El archivo contiene un ejemplo de un operador
erróneo en el contexto de una función.

\textbf{Resultado esperado:} Caso erróneo. 




\noindent
\textbf{Test \texttt{005-err-invalid-token.java}}

\textbf{Descripción:} El archivo contiene un ejemplo de un token 
inválido.

\textbf{Resultado esperado:} Caso erróneo.




\noindent
\textbf{Test \texttt{006-err-ampersand.java}}

\textbf{Descripción:} El archivo contiene un ejemplo de un token 
inválido.

\textbf{Resultado esperado:} Caso erróneo.




\noindent
\textbf{Test \texttt{007-err-comentario no finalizado.java}}

\textbf{Descripción:} El archivo contiene un ejemplo de un comentario
no cerrado.

\textbf{Resultado esperado:} Caso erróneo.




\noindent
\textbf{Test \texttt{008-cor-comments.java}}

\textbf{Descripción:} El archivo contiene código con varios ejemplos
de comentarios. La idea de este test es mostrar que aun con
comentarios, los números de linea y columna siguen siendo calculados
correctamente.

\textbf{Resultado esperado:} Caso correcto.




\noindent
\textbf{Test \texttt{009-err-invalid-lit-char.java}}

\textbf{Descripción:} El archivo contiene un ejemplo con un literal de
caractér inválido.

\textbf{Resultado esperado:} Caso erróneo.




\noindent
\textbf{Test \texttt{010-err-string-no-cerrado.java}}

\textbf{Descripción:} El archivo contiene un ejemplo con un string no
cerrado.

\textbf{Resultado esperado:} Caso erróneo.




\noindent
\textbf{Test \texttt{011-err-string-con-newline.java}}

\textbf{Descripción:} El archivo contiene un ejemplo con un string que
no esta cerrado en la misma linea.

\textbf{Resultado esperado:} Caso erróneo.




\noindent
\textbf{Test \texttt{012-intlit-malformado.java}}

\textbf{Descripción:} El archivo contiene un ejemplo con un literal de 
entero mal formado.

\textbf{Resultado esperado:} Caso erróneo.




\noindent
\textbf{Test \texttt{113-strlit-malformado.java}}

\textbf{Descripción:} El archivo contiene un ejemplo con un literal de
string mal formado.

\textbf{Resultado esperado:} Caso erróneo.




\noindent
\textbf{Test \texttt{114-charlit-malformado.java}}

\textbf{Descripción:} El archivo contiene un ejemplo con un litearl de 
caractér mal formado.

\textbf{Resultado esperado:} Caso erróneo.
\end{document}

